% This is a template for your written document.
%
% To compile using latexmk on the command line, run the following:
% latexmk -pdf main.tex

\documentclass[12pt]{article}
\usepackage{setspace}
\singlespace
\usepackage{graphicx}
\graphicspath{ {./imgs/} }
\usepackage[left=1in,right=1in,top=1in,bottom=1in]{geometry}

\title{\textbf{Leveraging Transformer-Based Sentiment Analysis for Financial Market Insights}}
\author{Anany Sachan}

\begin{document}

\maketitle

\section{Introduction}
Financial markets are profoundly influenced not only by economic fundamentals but also by the sentiments and
psychology of investors. In recent years, the proliferation of online financial news, social media platforms,
and discussion forums has created an abundance of textual data reflecting the real-time ``market mood''. This
has spurred growing interest in sentiment analysis as a tool to quantify and track these emotions and
opinions at scale. Major financial data providers now even offer sentiment indices as part of their
analytics services, underscoring the perceived value of such measures~\cite{10.1109/MCI.2018.2866727}. In
fact, the sentiment index of market participants has been extensively used for stock market prediction in
recent years~\cite{10.1109/MCI.2018.2866727}, with evidence that incorporating sentiment can improve
forecasting accuracy and investment decisions.

However, harnessing unstructured sentiment data effectively remains challenging due to the sheer volume and
velocity of text streams and the nuanced language (including slang, sarcasm, and domain-specific jargon)
prevalent in financial discourse. These challenges motivate the development of a real-time sentiment analysis
dashboard for finance---a system to ``nowcast'' market mood by continuously analyzing textual data sources.
\textit{Nowcasting}, in this context, refers to the real-time estimation of current market sentiment (as
opposed to traditional forecasting which predicts future trends). A reliable nowcasting tool could alert
traders and analysts to sudden shifts in investor optimism or fear, potentially offering early indicators of
market movements. For example, collective bullish enthusiasm on social media forums was a driving force
behind events such as the GameStop short squeeze in early 2021, where coordinated sentiment on Reddit's
WallStreetBets forum helped fuel extreme volatility in GameStop's stock price~\cite{Desiderio_2025}. This
project's significance lies in bridging advances in natural language processing with financial analytics to
capture such phenomena.

\section{Background}
\subsection{Natural Language Processing and Sentiment Analysis.}
\textit{Sentiment analysis} (or opinion mining) is a subfield of natural language processing (NLP) that
focuses on identifying and extracting subjective information from text---typically the polarity (positive,
negative, neutral) of opinions or the emotion and attitude expressed. Early approaches to sentiment analysis
often relied on lexicon-based methods: using dictionaries of sentiment-laden words to determine a text's
overall sentiment. For example, a simple lexicon-based system might count occurrences of ``positive'' words
minus ``negative'' words to assign a sentiment score. Domain-specific lexicons (such as Loughran and
McDonald's finance sentiment word lists) were developed to better handle financial terminology, which differs
from everyday language (e.g., words like ``bullish,'' ``bearish,'' or ``short'' have special meanings in
markets). While straightforward and interpretable, lexicon-based methods have inherent limitations---they
cannot easily account for context, sarcasm, negation, or shifting word usages, and their accuracy hinges on
the completeness of the predefined word list.

As the field matured, \textbf{machine learning} techniques quickly supplanted pure lexicon-based systems for
sentiment classification tasks. Instead of fixed dictionaries, machine learning approaches learn to infer
sentiment from examples of labeled text. A seminal work by Pang et al. (2002) showed that standard machine
learning classifiers (Naïve Bayes, maximum entropy, and support vector machines) significantly outperformed
human-crafted keyword baselines on movie review sentiment classification~\cite{10.3115/1118693.1118704}. This
study also highlighted that sentiment classification is more challenging than topic-based text
classification, owing to the need to detect subtle linguistic cues (for instance, negation or sarcasm) rather
than just content words~\cite{10.3115/1118693.1118704}. The implication was that more sophisticated features
and models were required to capture the nuance in sentiment-bearing language.

\subsection{Deep Learning and Transformer Models.}
In the past decade, advances in deep learning have dramatically improved the performance of NLP tasks,
including sentiment analysis. Neural network architectures like \textbf{convolutional neural networks (CNNs)}
and \textbf{recurrent neural networks (RNNs)} (particularly Long Short-Term Memory networks, LSTMs) enabled
models to automatically learn rich feature representations from text data. These approaches surpassed earlier
algorithms by capturing word order, semantic nuance, and contextual relationships more effectively than
bag-of-words models or manual feature engineering~\cite{10.1002/widm.1253}. Zhang et al.'s comprehensive
survey (2018) notes that deep learning methods began to consistently outperform traditional classifiers (such
as SVM or logistic regression) on sentiment tasks by learning multiple layers of abstract features~
\cite{10.1002/widm.1253}. The introduction of \textit{word embeddings} (e.g., Word2Vec, GloVe) also boosted
sentiment analysis, as words could be represented in vector spaces that encode semantic similarity, helping
algorithms generalize beyond exact keyword matches.

The most significant recent breakthrough has been the emergence of \textbf{transformer-based language models},
epitomized by \textbf{BERT} (Bidirectional Encoder Representations from Transformers) introduced by Devlin et
al.~\cite{devlin-etal-2019-bert}. Transformer models use self-attention mechanisms to capture long-range
dependencies in text and can be trained on massive corpora to learn contextual language representations.
Unlike earlier RNNs, transformers process words in parallel and consider both left and right context
simultaneously, enabling a deeper understanding of meaning. Fine-tuned transformer models now achieve
state-of-the-art results on a wide range of NLP benchmarks, including sentiment classification. For example,
one study found that a transformer-based classifier significantly outperformed an LSTM and other prior
models on a large Twitter sentiment dataset, particularly excelling at handling the noisy, informal language
of social media~\cite{10.1145/3650215.3650260}. The success of transformers has been so pronounced that they
have become the dominant paradigm in NLP, even spurring new directions such as \textit{multimodal} sentiment
analysis that combines text with audio or visual cues~\cite{10.1145/3586075}.

Despite these advancements, certain challenges in sentiment analysis persist. Subtle linguistic phenomena
like sarcasm, idioms, and context-dependent irony remain difficult for algorithms to fully grasp~
\cite{SentimentEmotionSurvey2021}. There are also concerns about biases in models (e.g., language models
learning biased associations from training data) and how well models generalize across domains~
\cite{SentimentEmotionSurvey2021}. Notably, the financial domain presents a unique context: language in
analyst reports, news headlines, or trader chats can be very domain-specific, filled with jargon and phrases
that are rare in general text corpora. A general-purpose model might misinterpret or simply not understand
such domain-specific language. This gap has led researchers to pursue domain adaptation strategies
---tailoring NLP models specifically for finance.

\subsection{FinBERT and Domain-Specific Modeling}
One pivotal development in this regard was the creation of \textbf{FinBERT} by Araci (2019)~
\cite{araci2019finbert}. FinBERT is a variant of the BERT model that was further pre-trained on large volumes
of financial text (e.g., news articles, earnings reports, and financial forums) to imbue it with
domain-specific knowledge. The motivation for FinBERT was that while generic BERT captures general language
patterns, it may struggle with specialized terminology and context found in finance (for instance,
interpreting ``market rally'' or ``dead cat bounce''). By continuing BERT's training on a finance corpus and
then fine-tuning it for sentiment classification, FinBERT achieved superior performance on financial
sentiment tasks compared to off-the-shelf models~\cite{araci2019finbert}.

In evaluations, FinBERT consistently outperformed general models and earlier deep learning methods when
classifying the sentiment of financial news and reports~\cite{araci2019finbert}. This demonstrated that
domain-specific language models can substantially improve accuracy by accounting for the nuances of
industry-specific language. FinBERT and similar finance-focused NLP models have since been widely adopted in
both research and industry for tasks like analyzing news sentiment, earnings call transcripts, and social
media discussions related to stocks.

In summary, the evolution of sentiment analysis techniques---from lexicon approaches to machine learning, and
from simple classifiers to transformers like BERT---has provided an expanding toolkit for tackling the
problem of understanding market mood. The challenge now lies in applying these tools effectively to real-time
financial data streams, which is the focus of our project.

\section{Related Work}
Research at the intersection of textual sentiment analysis and finance is rich and multifaceted. Broadly,
prior work can be grouped into two themes: (1) developing specialized sentiment analysis models for financial
language, and (2) applying sentiment-based indicators to financial forecasting or market analysis.

\subsection{Financial Sentiment Models}
A cornerstone in this area is the aforementioned \textbf{FinBERT} model.
Araci's work~\cite{araci2019finbert} demonstrated that adapting a transformer to financial text data yields
clear benefits for sentiment classification in finance. FinBERT's introduction has spurred further research
into domain-specific NLP, and it serves as a foundation for many subsequent studies. For instance, Jiang and
Zeng~\cite{jiang2025financialsentiment} leverage FinBERT to extract sentiment signals from financial news,
which they then input into a predictive model for stock movement. In their approach, daily news articles are
fed through FinBERT to produce sentiment scores or embeddings, and these features are used alongside an
LSTM-based temporal model to forecast stock price trends. They report that the FinBERT-enhanced model
significantly outperforms comparable models using a generic BERT or using no text input at all, confirming
that finance-tailored language models can improve predictive accuracy in market tasks. This finding aligns
with the general intuition that more informative representations of text (in this case, capturing
finance-specific context) translate into better downstream predictions.

\subsection{Sentiment in Market Prediction}
Even before the deep learning era, researchers explored links between public sentiment and market behavior.
A variety of textual sources have been studied, including news, financial reports, and
social media. Early studies (e.g., 2000s-era works by Tetlock and others) found that negative tone
in news or investor forums can predict short-term dips in stock prices, suggesting that sentiment contains
predictive signal. More recent work has continued to validate and extend these insights. Xing et al. (2018)
provide a clear example by constructing a sentiment index from social media posts and integrating it into an
asset allocation framework~\cite{10.1109/MCI.2018.2866727}. They use an ensemble of clustering and LSTM
models to process streams of Twitter and forum data, distill a market sentiment time series, and incorporate
it as ``market views'' in a Bayesian portfolio optimization. The result was improved portfolio performance
(in terms of stability and returns) compared to strategies that ignore sentiment~
\cite{10.1109/MCI.2018.2866727}. This study not only underscores that sentiment can enhance predictive
models, but also illustrates a methodology for merging textual signals with traditional financial theories
(Modern Portfolio Theory, in that case). Similarly, other works have used sentiment extracted from news
headlines or financial blogs to forecast stock returns or volatility, often reporting that sentiment features
add incremental predictive power on top of technical or fundamental features. The consensus emerging from
these studies is that there is measurable information content in the collective mood of market participants,
which, if quantified correctly, can be useful for nowcasting and forecasting financial market dynamics.

\subsection{Social Media and Alternative Data}
A particularly vibrant strand of recent research focuses on social media sentiment, given the
outsized impact platforms like Twitter, Reddit, and StockTwits now have on retail investor behavior. Social
media data is noisy and rife with slang, memes, and unstructured narratives, making it challenging for
traditional NLP models.

Deng et al. (2023) highlight this challenge in their Reddit sentiment analysis study:
they note that understanding content from the r/WallStreetBets community requires both financial knowledge
and fluency in internet vernacular, which makes obtaining high-quality labeled data difficult~
\cite{10.1145/3543873.3587605}. To tackle this, they employ a \textbf{semi-supervised learning} pipeline
using a large language model (GPT-3 variant) to generate ``weak'' sentiment labels for thousands of Reddit
posts. These LLM-generated labels (refined through prompt techniques like chain-of-thought reasoning) are
then used to train a smaller, deployable model~\cite{10.1145/3543873.3587605}. Remarkably, with only a
handful of manual prompts to guide the LLM, the final distilled model achieved accuracy on par with fully
supervised models---illustrating the great potential of LLMs to bootstrap sentiment analysis when
human-labeled data is scarce.

This approach is especially relevant for finance, where new slang (e.g., ``diamond hands,'' ``to the moon'')
and rapidly evolving topics can quickly outdated static lexicons or past training data. In addition to
methodological advances, social media has also been the subject of case studies linking sentiment to market
events. For example, Desiderio et al. (2025) examine the dynamics of the Reddit-driven GameStop short
squeeze, quantitatively analyzing how collective bullish sentiment online coalesced into a coordinated buying
frenzy~\cite{Desiderio_2025}. Their findings shed light on the feedback loop between viral social-media
sentiment and extreme market outcomes, reinforcing why real-time monitoring of such sentiment is important.

\section{Methodology}
Our primary goal is to create a real-time sentiment analysis dashboard that can nowcast market mood by
aggregating textual data from multiple sources. In particular, we focus on two contrasting data streams: (1)
financial news articles, which provide formal and vetted information, and (2) social media posts from
platforms like Reddit, which capture informal investor opinion. These sources were chosen to balance
credibility and timeliness - news reflects the perspectives of journalists and analysts, while Reddit (e.g.,
posts from communities such as \texttt{r/WallStreetBets}) reflects retail investor sentiment in the wild.
Both have been identified in prior research as rich sources of market sentiment
\cite{10.1109/MCI.2018.2866727}. The target sentiment categories in our analysis are positive, negative, or
neutral, following standard practice in sentiment classification. Each incoming text will be labeled with one
of these polarities. We employ the transformer-based FinBERT model for this purpose, as it has been
pre-trained and fine-tuned specifically for financial language~\cite{araci2019finbert}, making it well-suited
to detect nuances in finance-specific text (e.g., distinguishing bullish vs. bearish context).

\begin{figure}[h!]
  \centering
  \includegraphics[width=0.9\textwidth]{system-logic.png}
  \caption{System architecture for the Dashboard}
  \label{fig:arch}
\end{figure}

At a high level, our system's architecture follows a pipeline from user input to sentiment output.
Figure~\ref{fig:arch} illustrates this overall architecture. The process can be summarized in several key
stages:

\begin{enumerate}
  \item \textbf{User Input \& Query Formation:} The user selects or inputs a target of interest (such as a
        specific stock ticker, company name, or market topic). This query defines the context for sentiment
        analysis, ensuring that only relevant data is collected (e.g., news articles about that company, or
        Reddit posts mentioning the ticker).
  \item \textbf{Data Ingestion from Multiple Sources:} For each query, the system concurrently fetches new
        textual data from the chosen sources. A news adapter retrieves the latest relevant news articles
        (e.g., headlines or snippets from financial news APIs or RSS feeds), and a social media adapter pulls
        recent Reddit posts or comments from finance-related subreddits that mention the query. We throttle
        and schedule these requests to effectively capture real-time information without overloading any
        single source.
  \item \textbf{Text Preprocessing:} The collected texts are then cleaned and preprocessed. This step
        includes removing irrelevant content (such as HTML tags, URLs, or special characters), normalizing
        text (lowercasing, handling common slang or abbreviations), and potentially filtering out very short
        or non-English messages. Domain-specific tokens (e.g., ticker symbols like GME, or slang like ``HODL'')
        are preserved since they carry information. The goal is to format each text into a clean input string
        suitable for the language model.
  \item \textbf{Sentiment Classification with FinBERT:} Each preprocessed text is fed into the FinBERT model
        to predict its sentiment category. FinBERT's output includes probabilities for the three classes,
        which we convert into a discrete label (positive/negative/neutral) for simplicity. Because FinBERT is
        trained on financial corpora, it can interpret context that generic models might misclassify; for
        example, it understands that a sentence like “Dow falls on fears of rate hike” should be read as a
        negative sentiment regarding market outlook. We do not further fine-tune FinBERT in this project (to
        avoid expensive training), relying on the model's off-the-shelf capabilities~\cite{araci2019finbert}.
        Each piece of text now has an associated sentiment score or label.
  \item \textbf{Aggregation and Sentiment Index Computation:} Finally, the system aggregates individual
        sentiments into an overall sentiment indicator. Depending on the use case, this could be a simple
        aggregate (such as the percentage of positive vs. negative items in the last hour) or a weighted
        sentiment score. In our implementation, we compute a sentiment index by assigning +1 to positive, -1
        to negative (and 0 to neutral) classifications and averaging over all texts within a given time
        window. This yields a continuous sentiment signal that can be tracked over time. The index can be
        visualized as a time series to show sentiment trends (e.g., a sharp drop in the index might indicate
        a surge of pessimistic news and posts). In addition, separate sentiment tallies for each source are
        maintained so that the dashboard can display source-specific sentiments side by side.
\end{enumerate}

Through this pipeline, the system effectively transforms unstructured text streams into quantifiable
sentiment metrics in real-time. The methodology ensures that our sentiment analysis is context-aware (thanks
to FinBERT's domain knowledge) and comprehensive in coverage (by drawing from both news media and social
media). In the next section, we discuss how this methodology was realized in practice via our system's
implementation.

\section{Implementation}
To implement the above approach, we developed a modular web-based application consisting of a Python backend
and a JavaScript frontend. The backend is built with the lightweight Flask framework, which exposes a RESTful
API for data retrieval and sentiment analysis. The frontend is built in React, creating an interactive
single-page application that displays the sentiment dashboard. This separation of concerns allows the heavy
NLP processing to occur server-side, while users interact with a smooth, responsive interface in their web
browser. The overall software architecture is service-oriented: the Flask API endpoints serve JSON data that
the React frontend fetches and renders. Figure~\ref{fig:ui} shows a prototype of the user
interface, including charts for sentiment trends and controls for the user to select different stock tickers or time ranges.

On the backend, we implemented dedicated ingestion adapters for each data source.
\textit{Expand this later when adapters are implemented}.

Both adapters feed incoming text to a preprocessing pipeline. Implemented in Python, this pipeline cleans and
normalizes text as described in the methodology. Specifically, we remove URLs, user mentions, and markup from
Reddit text (while preserving salient tokens like cashtags or tickers that start with \$). We also strip
common stop words and punctuation from news snippets to reduce clutter for the language model. In this stage,
we pay attention to case normalization (since FinBERT is cased, we generally preserve case), and we ensure
that each text does not exceed BERT's maximum token length (512 tokens). If a news article is too long (which
is uncommon for headlines/snippets), it is truncated or split into multiple segments for analysis.

After preprocessing, the core sentiment classification step is handled by FinBERT. We integrate FinBERT using
the Hugging Face Transformers library, loading a pre-trained FinBERT model fine-tuned for sentiment analysis.
The model is loaded once when the server starts and remains in memory to serve predictions quickly on demand.
For each cleaned text, we call the model's inference pipeline (tokenizing the text and obtaining the softmax
probabilities for each sentiment class). We then assign a sentiment label based on the highest probability.
This entire process is encapsulated in a Flask route (for example, a \texttt{/analyze} endpoint) so that the
front-end can trigger sentiment analysis via an HTTP request.

On the frontend side, the React dashboard polls these API endpoints to get the newest data. The interface is
designed to present key information at a glance. It includes: (a) a time-series chart that plots the
sentiment index over time, updating in near real-time (e.g., a new point every minute), (b) a bar chart or
pie chart showing the proportion of positive/negative/neutral posts in the latest window (with different
colors for news vs. social media contributions), and (c) controls for the user to change the query or adjust
settings (such as selecting a different subreddit or news source, or pausing the live updates).
Figure~\ref{fig:ui} provides a snapshot of the dashboard prototype, where the sentiment index trend line is
shown alongside a table of recent high-sentiment headlines. The UI emphasizes clarity and immediacy, using
visual cues (green for positive, red for negative) to highlight sentiment changes. We also implement basic
error handling and status indicators (for example, if an API call fails or if no data is available for a
given query, the UI displays a message so the user is aware of the system state).

Throughout the implementation, a priority was placed on modularity and extensibility. New data sources (for
example, Twitter or StockTwits) could be added by writing a new adapter that conforms to the same interface.
Likewise, the sentiment model can be swapped out or upgraded (for instance, if a newer finance-specific LLM
becomes available) with minimal changes to the surrounding code, thanks to the abstraction of the
\texttt{/analyze} API route. The use of Flask and React for the prototype proved sufficient for our needs;
Flask can comfortably handle the moderate request load of our polling mechanism, and React enables dynamic
updates and interactive visualizations in the browser.

\begin{figure}[h]
  \centering
  \textit{Placeholder figure to be replaced with an actual screenshot}
  \caption{Prototype of the web-based dashboard UI. The interface includes (left) a line chart showing the
    computed sentiment index over time for a selected stock, and (right) a summary of recent
    posts/news with their individual sentiment classifications (green for positive, red for negative,
    gray for neutral). Users can input different tickers or topics to refresh the analysis.}
  \label{fig:ui}
\end{figure}


\section{Results}
At this stage of the project, data collection and integration are ongoing, so we present an outline of the
expected results and how we plan to analyze them. The evaluation of our system will be primarily qualitative
and exploratory, given that we do not have a labeled “ground truth” for real-time market sentiment. Key
results will revolve around visualizing sentiment trends, comparing sources, and assessing the FinBERT
model's outputs. We anticipate the following findings and will include corresponding charts and tables in
the final report:

\begin{enumerate}
  \item \textbf{Sentiment Trend Visualization:} We will present time-series plots of the computed sentiment
        index for a particular financial instrument or topic over various time windows (e.g., intraday,
        daily, weekly). For example, Figure~\ref{fig:timeline} will show the sentiment index for a chosen
        stock (say, Tesla) over a multi-day period. These charts will illustrate how sentiment fluctuates in
        response to market events, news releases, or social media activity. For example, we expect to see
        spikes in negative sentiment during market sell-offs or positive sentiment following favorable
        earnings reports. By examining these patterns, we can demonstrate the system's ability to capture
        market mood dynamics. In the final thesis, we plan to highlight a couple of specific case studies
        where the sentiment timeline clearly reflected an important market narrative, thereby showcasing the
        dashboard's practical relevance.

        \begin{figure}[h!]
          \centering
          \textit{Placeholder for sentiment index timeline visualization}
          \caption{Example sentiment index timeline for a selected stock over one week. The y-axis
            represents the aggregated sentiment score (positive or negative tilt), and the x-axis is time
            (days). Notable external events (earnings release, major news) are annotated to show alignment
            with sentiment spikes.}
          \label{fig:timeline}
        \end{figure}

  \item \textbf{Source Comparison Analysis:} We will analyze and compare sentiment distributions
        between news articles and social media posts. Table~\ref{tab:source-breakdown} illustrates the kind
        of summary we will provide. We hypothesize that social media may exhibit more extreme sentiment
        swings (due to retail investor enthusiasm or panic) compared to the more measured tone of news media.
        Statistical summaries (e.g., mean sentiment scores, variance) will quantify these differences.

        \begin{table}[h!]
          \centering
          \textit{Placeholder for source-wise sentiment breakdown table}
          \caption{This table will list, for a given period, the number of items from each source (News vs. Reddit),
          and the percentage of those items labeled as Positive, Neutral, or Negative by FinBERT. It will
          also include the average sentiment score per source. An example (for illustration only) might be:
          {\small News: 40 articles (15\% pos, 70\% neutral, 15\% neg, avg score = 0.0); Reddit: 120 posts
          (25\% pos, 50\% neutral, 25\% neg, avg score = 0.0).}}
          \label{tab:source-breakdown}
        \end{table}

  \item \textbf{Model Output Assessment:} We will examine the reliability and characteristics of the FinBERT
        model's outputs on our collected data. Although we cannot directly measure accuracy without labeled
        test data, we can analyze proxies for reliability. One analysis will involve checking the
        distribution of predicted labels. If we find, for instance, that an overwhelmingly high percentage of
        items are labeled “Neutral,” it might indicate that FinBERT is defaulting to neutral in uncertain
        cases (which could dampen the sensitivity of our index). Conversely, a balanced or contextually
        appropriate distribution (e.g., more negatives on a day of bad news) would build confidence in the
        model's usefulness. We may include a chart showing the proportion of each sentiment class in a
        volatile period versus a calm period to illustrate that the model's output varies with market
        conditions (which it should if it's capturing real sentiment).\newline
        Additionally, we plan a qualitative assessment: manually reading a sample of texts alongside the
        model's label to subjectively judge correctness. Early informal tests suggest FinBERT is quite adept
        with straightforward cases (it correctly labels headlines like ``Stocks soar after earnings beat'' as
        positive, and ``Company X faces fraud allegations'' as negative). However, it can be challenged by
        sarcasm or figurative language in social posts - a known difficulty for sentiment models
        \cite{SentimentEmotionSurvey2021}. We will document any systematic errors observed (for example,
        finance slang that confuses the model, or instances where context was missed) and may update our
        preprocessing or handling of model confidence accordingly.
\end{enumerate}

\section{Discussion}
The discussion will interpret our findings and position them in the context of prior work and the broader
goals of the project. Assuming the results align with expectations, several points of analysis will be
explored:

\begin{enumerate}
  \item \textbf{Relationship to prior work:}
  \item \textbf{System Strengths:}
  \item \textbf{Implications for investors and analysts:}
  \item \textbf{Limitations of the current system:}
  \item \textbf{Potential improvements and future directions:}
\end{enumerate}

In conclusion to the discussion, we will tie everything back to the thesis statement: our real-time sentiment
dashboard demonstrates a practical fusion of NLP and finance, highlights the value of domain-specific models
like FinBERT in extracting actionable insight from text, and offers a foundation that future work can build
upon to deepen the integration of alternative data (like textual sentiment) into market analysis tools.9090

\section{Conclusion}
In summary, this study has developed a framework for leveraging transformer-based sentiment analysis in the
context of financial markets. We designed and implemented a real-time dashboard that aggregates sentiment
from both news media and social media, providing a nuanced “market mood” indicator for investors and
analysts. By utilizing FinBERT, a language model tailored to finance, our system can understand
domain-specific chatter and news with greater fidelity than generic sentiment tools. The expected outcome is
a tool that not only visualizes sentiment trends in an intuitive way but also bridges a gap between
cutting-edge NLP techniques and practical financial analytics.

This work contributes an integrated approach that marries multiple data sources and advanced AI in a
real-time application. The modular design means it can be continually updated: new data feeds and improved
models can be incorporated as the landscape evolves. In its current form, the dashboard serves as a
proof-of-concept for how unstructured textual data might be harnessed for market insight, aligning with a
growing industry trend of using alternative data in finance.

For future work, there are several promising directions. One immediate extension would be to expand the
coverage of the system - for example, including Twitter data or transcripts of earnings calls to capture even
more facets of market sentiment. Another avenue is to refine the sentiment analysis itself, perhaps by
fine-tuning the model on a larger, more recent corpus of financial discussions (or employing prompt-based
large language models for greater nuance). There is also potential to quantitatively evaluate the predictive
value of the sentiment index by correlating it with subsequent market movements, thus moving from nowcasting
towards forecasting. Ultimately, we envision that efforts like this project can evolve into robust tools that
assist investors by flagging early signs of optimism or fear in the market. In conclusion, the project
underscores the value of interdisciplinary approaches - combining NLP, data engineering, and finance - to
innovate in understanding and visualizing the complex emotional undercurrents that drive economic behavior.
We hope this study lays groundwork for more sophisticated sentiment-driven analytics in the financial domain,
and we have outlined several improvements that can be pursued to enhance both the accuracy and utility of
such systems in the future.


\bibliographystyle{acm}
\bibliography{bibliography.bib}

\end{document}
