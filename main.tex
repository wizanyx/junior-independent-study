% This is a template for your written document.
%
% To compile using latexmk on the command line, run the following:
% latexmk -pdf main.tex

\documentclass[12pt]{article}
\usepackage{setspace}
\singlespace
\usepackage[left=1in,right=1in,top=1in,bottom=1in]{geometry}

\title{\textbf{Leveraging Transformer-Based Sentiment Analysis for Financial Market Insights}}
\author{Anany Sachan}

\begin{document}

\maketitle

\section{Introduction}
Financial markets are profoundly influenced not only by economic fundamentals but also by the sentiments and
psychology of investors. In recent years, the proliferation of online financial news, social media platforms,
and discussion forums has created an abundance of textual data reflecting the real-time ``market mood''. This
has spurred growing interest in sentiment analysis as a tool to quantify and track these emotions and
opinions at scale. Major financial data providers now even offer sentiment indices as part of their
analytics services, underscoring the perceived value of such measures~\cite{10.1109/MCI.2018.2866727}. In
fact, the sentiment index of market participants has been extensively used for stock market prediction in
recent years~\cite{10.1109/MCI.2018.2866727}, with evidence that incorporating sentiment can improve
forecasting accuracy and investment decisions.

However, harnessing unstructured sentiment data effectively remains challenging due to the sheer volume and
velocity of text streams and the nuanced language (including slang, sarcasm, and domain-specific jargon)
prevalent in financial discourse. These challenges motivate the development of a real-time sentiment analysis
dashboard for finance---a system to ``nowcast'' market mood by continuously analyzing textual data sources.
\textit{Nowcasting}, in this context, refers to the real-time estimation of current market sentiment (as
opposed to traditional forecasting which predicts future trends). A reliable nowcasting tool could alert
traders and analysts to sudden shifts in investor optimism or fear, potentially offering early indicators of
market movements. For example, collective bullish enthusiasm on social media forums was a driving force
behind events such as the GameStop short squeeze in early 2021, where coordinated sentiment on Reddit's
WallStreetBets forum helped fuel extreme volatility in GameStop's stock price~\cite{Desiderio_2025}. This
project's significance lies in bridging advances in natural language processing with financial analytics to
capture such phenomena.

\section{Background}
\subsection{Natural Language Processing and Sentiment Analysis.}
\textit{Sentiment analysis} (or opinion mining) is a subfield of natural language processing (NLP) that
focuses on identifying and extracting subjective information from text---typically the polarity (positive,
negative, neutral) of opinions or the emotion and attitude expressed. Early approaches to sentiment analysis
often relied on lexicon-based methods: using dictionaries of sentiment-laden words to determine a text's
overall sentiment. For example, a simple lexicon-based system might count occurrences of ``positive'' words
minus ``negative'' words to assign a sentiment score. Domain-specific lexicons (such as Loughran and
McDonald's finance sentiment word lists) were developed to better handle financial terminology, which differs
from everyday language (e.g., words like ``bullish,'' ``bearish,'' or ``short'' have special meanings in
markets). While straightforward and interpretable, lexicon-based methods have inherent limitations---they
cannot easily account for context, sarcasm, negation, or shifting word usages, and their accuracy hinges on
the completeness of the predefined word list.

As the field matured, \textbf{machine learning} techniques quickly supplanted pure lexicon-based systems for
sentiment classification tasks. Instead of fixed dictionaries, machine learning approaches learn to infer
sentiment from examples of labeled text. A seminal work by Pang et al. (2002) showed that standard machine
learning classifiers (Naïve Bayes, maximum entropy, and support vector machines) significantly outperformed
human-crafted keyword baselines on movie review sentiment classification~\cite{10.3115/1118693.1118704}. This
study also highlighted that sentiment classification is more challenging than topic-based text
classification, owing to the need to detect subtle linguistic cues (for instance, negation or sarcasm) rather
than just content words~\cite{10.3115/1118693.1118704}. The implication was that more sophisticated features
and models were required to capture the nuance in sentiment-bearing language.

\subsection{Deep Learning and Transformer Models.}
In the past decade, advances in deep learning have dramatically improved the performance of NLP tasks,
including sentiment analysis. Neural network architectures like \textbf{convolutional neural networks (CNNs)}
and \textbf{recurrent neural networks (RNNs)} (particularly Long Short-Term Memory networks, LSTMs) enabled
models to automatically learn rich feature representations from text data. These approaches surpassed earlier
algorithms by capturing word order, semantic nuance, and contextual relationships more effectively than
bag-of-words models or manual feature engineering~\cite{10.1002/widm.1253}. Zhang et al.'s comprehensive
survey (2018) notes that deep learning methods began to consistently outperform traditional classifiers (such
as SVM or logistic regression) on sentiment tasks by learning multiple layers of abstract features~
\cite{10.1002/widm.1253}. The introduction of \textit{word embeddings} (e.g., Word2Vec, GloVe) also boosted
sentiment analysis, as words could be represented in vector spaces that encode semantic similarity, helping
algorithms generalize beyond exact keyword matches.

The most significant recent breakthrough has been the emergence of \textbf{transformer-based language models},
epitomized by \textbf{BERT} (Bidirectional Encoder Representations from Transformers) introduced by Devlin et
al.~\cite{devlin-etal-2019-bert}. Transformer models use self-attention mechanisms to capture long-range
dependencies in text and can be trained on massive corpora to learn contextual language representations.
Unlike earlier RNNs, transformers process words in parallel and consider both left and right context
simultaneously, enabling a deeper understanding of meaning. Fine-tuned transformer models now achieve
state-of-the-art results on a wide range of NLP benchmarks, including sentiment classification. For example,
one study found that a transformer-based classifier significantly outperformed an LSTM and other prior
models on a large Twitter sentiment dataset, particularly excelling at handling the noisy, informal language
of social media~\cite{10.1145/3650215.3650260}. The success of transformers has been so pronounced that they
have become the dominant paradigm in NLP, even spurring new directions such as \textit{multimodal} sentiment
analysis that combines text with audio or visual cues~\cite{10.1145/3586075}.

Despite these advancements, certain challenges in sentiment analysis persist. Subtle linguistic phenomena
like sarcasm, idioms, and context-dependent irony remain difficult for algorithms to fully grasp~
\cite{SentimentEmotionSurvey2021}. There are also concerns about biases in models (e.g., language models
learning biased associations from training data) and how well models generalize across domains~
\cite{SentimentEmotionSurvey2021}. Notably, the financial domain presents a unique context: language in
analyst reports, news headlines, or trader chats can be very domain-specific, filled with jargon and phrases
that are rare in general text corpora. A general-purpose model might misinterpret or simply not understand
such domain-specific language. This gap has led researchers to pursue domain adaptation strategies
---tailoring NLP models specifically for finance.

\subsection{FinBERT and Domain-Specific Modeling}
One pivotal development in this regard was the creation of \textbf{FinBERT} by Araci (2019)~
\cite{araci2019finbert}. FinBERT is a variant of the BERT model that was further pre-trained on large volumes
of financial text (e.g., news articles, earnings reports, and financial forums) to imbue it with
domain-specific knowledge. The motivation for FinBERT was that while generic BERT captures general language
patterns, it may struggle with specialized terminology and context found in finance (for instance,
interpreting ``market rally'' or ``dead cat bounce''). By continuing BERT's training on a finance corpus and
then fine-tuning it for sentiment classification, FinBERT achieved superior performance on financial
sentiment tasks compared to off-the-shelf models~\cite{araci2019finbert}.

In evaluations, FinBERT consistently outperformed general models and earlier deep learning methods when
classifying the sentiment of financial news and reports~\cite{araci2019finbert}. This demonstrated that
domain-specific language models can substantially improve accuracy by accounting for the nuances of
industry-specific language. FinBERT and similar finance-focused NLP models have since been widely adopted in
both research and industry for tasks like analyzing news sentiment, earnings call transcripts, and social
media discussions related to stocks.

In summary, the evolution of sentiment analysis techniques---from lexicon approaches to machine learning, and
from simple classifiers to transformers like BERT---has provided an expanding toolkit for tackling the
problem of understanding market mood. The challenge now lies in applying these tools effectively to real-time
financial data streams, which is the focus of our project.

\section{Related Work}
Research at the intersection of textual sentiment analysis and finance is rich and multifaceted. Broadly,
prior work can be grouped into two themes: (1) developing specialized sentiment analysis models for financial
language, and (2) applying sentiment-based indicators to financial forecasting or market analysis.

\subsection{Financial Sentiment Models}
A cornerstone in this area is the aforementioned \textbf{FinBERT} model.
Araci's work~\cite{araci2019finbert} demonstrated that adapting a transformer to financial text data yields
clear benefits for sentiment classification in finance. FinBERT's introduction has spurred further research
into domain-specific NLP, and it serves as a foundation for many subsequent studies. For instance, Jiang and
Zeng~\cite{jiang2025financialsentiment} leverage FinBERT to extract sentiment signals from financial news,
which they then input into a predictive model for stock movement. In their approach, daily news articles are
fed through FinBERT to produce sentiment scores or embeddings, and these features are used alongside an
LSTM-based temporal model to forecast stock price trends. They report that the FinBERT-enhanced model
significantly outperforms comparable models using a generic BERT or using no text input at all, confirming
that finance-tailored language models can improve predictive accuracy in market tasks. This finding aligns
with the general intuition that more informative representations of text (in this case, capturing
finance-specific context) translate into better downstream predictions.

\subsection{Sentiment in Market Prediction}
Even before the deep learning era, researchers explored links between public sentiment and market behavior.
A variety of textual sources have been studied, including news, financial reports, and
social media. Early studies (e.g., 2000s-era works by Tetlock and others) found that negative tone
in news or investor forums can predict short-term dips in stock prices, suggesting that sentiment contains
predictive signal. More recent work has continued to validate and extend these insights. Xing et al. (2018)
provide a clear example by constructing a sentiment index from social media posts and integrating it into an
asset allocation framework~\cite{10.1109/MCI.2018.2866727}. They use an ensemble of clustering and LSTM
models to process streams of Twitter and forum data, distill a market sentiment time series, and incorporate
it as ``market views'' in a Bayesian portfolio optimization. The result was improved portfolio performance
(in terms of stability and returns) compared to strategies that ignore sentiment~
\cite{10.1109/MCI.2018.2866727}. This study not only underscores that sentiment can enhance predictive
models, but also illustrates a methodology for merging textual signals with traditional financial theories
(Modern Portfolio Theory, in that case). Similarly, other works have used sentiment extracted from news
headlines or financial blogs to forecast stock returns or volatility, often reporting that sentiment features
add incremental predictive power on top of technical or fundamental features. The consensus emerging from
these studies is that there is measurable information content in the collective mood of market participants,
which, if quantified correctly, can be useful for nowcasting and forecasting financial market dynamics.

\subsection{Social Media and Alternative Data}
A particularly vibrant strand of recent research focuses on social media sentiment, given the
outsized impact platforms like Twitter, Reddit, and StockTwits now have on retail investor behavior. Social
media data is noisy and rife with slang, memes, and unstructured narratives, making it challenging for
traditional NLP models.

Deng et al. (2023) highlight this challenge in their Reddit sentiment analysis study:
they note that understanding content from the r/WallStreetBets community requires both financial knowledge
and fluency in internet vernacular, which makes obtaining high-quality labeled data difficult~
\cite{10.1145/3543873.3587605}. To tackle this, they employ a \textbf{semi-supervised learning} pipeline
using a large language model (GPT-3 variant) to generate ``weak'' sentiment labels for thousands of Reddit
posts. These LLM-generated labels (refined through prompt techniques like chain-of-thought reasoning) are
then used to train a smaller, deployable model~\cite{10.1145/3543873.3587605}. Remarkably, with only a
handful of manual prompts to guide the LLM, the final distilled model achieved accuracy on par with fully
supervised models---illustrating the great potential of LLMs to bootstrap sentiment analysis when
human-labeled data is scarce.

This approach is especially relevant for finance, where new slang (e.g., ``diamond hands,'' ``to the moon'')
and rapidly evolving topics can quickly outdated static lexicons or past training data. In addition to
methodological advances, social media has also been the subject of case studies linking sentiment to market
events. For example, Desiderio et al. (2025) examine the dynamics of the Reddit-driven GameStop short
squeeze, quantitatively analyzing how collective bullish sentiment online coalesced into a coordinated buying
frenzy~\cite{Desiderio_2025}. Their findings shed light on the feedback loop between viral social-media
sentiment and extreme market outcomes, reinforcing why real-time monitoring of such sentiment is important.

\section{Methodology}

\section{Results}

\section{Conclusion}

\bibliographystyle{acm}
\bibliography{bibliography.bib}

\end{document}
