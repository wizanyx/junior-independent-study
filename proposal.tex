\documentclass[12pt]{article}
\usepackage{setspace}
\singlespace
\usepackage{graphicx}
\graphicspath{ {./imgs/} }
\usepackage[left=1in,right=1in,top=1in,bottom=1in]{geometry}

\title{\textbf{Leveraging Transformer-Based Sentiment Analysis for Financial Market Insights}}
\author{Anany Sachan}

\begin{document}

\maketitle

In recent decades, sentiment analysis has emerged as one of the most widely studied applications of
natural language processing (NLP). At its core, sentiment analysis refers to the automatic detection of
opinions, attitudes, and emotions expressed in text. Foundational work by Pang, Lee, and Vaithyanathan
established that classical machine learning methods such as Naïve Bayes, Maximum Entropy, and Support
Vector Machines could be successfully applied to classify movie reviews as positive or negative
\cite{10.3115/1118693.1118704}. This research established sentiment classification as a formal
computational task and inspired a wave of subsequent studies across domains ranging from politics to
product reviews.

As the field matured, researchers began to move beyond polarity detection toward the identification of
richer affective states. Mohammad surveyed advances in detecting valence, discrete emotions such as joy,
anger, or fear, and other affectual categories, while also highlighting issues of bias, fairness, and
societal consequences \cite{SentimentEmotionSurvey2021}. Around the same time, deep learning models
began to dominate sentiment analysis tasks. Surveys such as Zhang, Wang, and Liu documented how
convolutional and recurrent neural networks dramatically improved performance compared to traditional
approaches, especially when trained on large labeled datasets \cite{10.1002/widm.1253}.

The most significant methodological breakthrough came with the introduction of transformer-based
language models, particularly BERT and its domain-specific variants. Transformers rely on attention
mechanisms that capture long-range dependencies in text, enabling much more accurate contextual
representations than bag-of-words or recurrent architectures. Subsequent surveys and applied studies
demonstrated that transformer models outperform previous baselines across a wide range of sentiment
tasks \cite{10.1145/3586075,10.1145/3650215.3650260}. In the financial domain, fine-tuned transformer
models such as FinBERT have been shown to significantly improve sentiment detection in technical texts
like earnings reports and analyst briefings \cite{10.1145/3543873.3587605,araci2019finbert}.

The significance of these advances lies in the growing recognition that public sentiment is a powerful
driver of financial markets. Traditional quantitative models, which rely primarily on historical prices
and trading volume, fail to account for the behavioral and psychological forces that often dictate
short-term volatility. Events such as the 2021 GameStop short squeeze—fueled largely by sentiment on
Reddit's r/wallstreetbets—illustrate the real-world consequences of collective investor mood
\cite{Desiderio_2025}. Incorporating sentiment signals into financial analysis has the potential to
improve forecasting, risk management, and decision-making \cite{jiang2025financialsentiment}. Recent
work has begun to test this hypothesis, showing correlations between sentiment derived from social
media or news and subsequent asset price movements \cite{10.1109/MCI.2018.2866727}. However, these
methods remain far from standardized, and questions remain regarding reliability, data bias, and
robustness across market conditions.

The goal of this project is to design and implement a finance-focused sentiment analysis system that
builds on the trajectory of prior research while addressing gaps in applied usage. Specifically, I will
develop a \textbf{Finance Sentiment Dashboard}: a software application that ingests financial text data
(news headlines, social media posts, or discussion forum content) and outputs sentiment classifications
and visualizations in near real-time. The system will leverage a transformer-based model fine-tuned for
financial text, such as FinBERT, to provide ticker-specific sentiment scores and trends.

The proposed software will have three main capabilities. First, users will be able to enter a stock
ticker symbol, at which point the backend will collect relevant recent text data from configured
sources. Second, the system will preprocess the data and apply the sentiment model to classify texts
as positive, negative, or neutral with respect to financial outlook. Third, the frontend will display
results in a clear, interactive dashboard, including aggregate sentiment metrics, temporal trend charts,
and confidence estimates. The pipeline flow is illustrated in Figure \ref{fig:system-logic}.

\begin{figure}[h!]
    \centering
    \includegraphics[width=0.9\textwidth]{system-logic.png}
    \caption{System architecture for the Finance Sentiment Dashboard.}
    \label{fig:system-logic}
\end{figure}

By combining academic rigor with practical functionality, this project will contribute both to scholarly
discussions and to applied finance practice. It demonstrates how the evolution of sentiment analysis —
from early machine learning approaches \cite{10.3115/1118693.1118704}, through emotional nuance
\cite{SentimentEmotionSurvey2021}, to modern transformer architectures
\cite{10.1145/3586075,10.1145/3543873.3587605} — can be operationalized into a real-world tool. At the
same time, it aims to provide users with an accessible interface for exploring how public mood
influences financial markets, reinforcing the significance of sentiment as both a computational and
economic phenomenon.

\newpage
\section*{Appendix}
A concise list of features / user stories in the order in which they will be built.

\begin{itemize}
    \item Set up the foundational project environment for backend (Flask) and frontend (React),
          including code quality tools, base tests, and optional CI workflow.
    \item Configure environment variables and secrets for API keys and external service access.
    \item Define and enforce a common document schema for all ingested and uploaded texts across
          adapters and APIs.
    \item Implement a preprocessing pipeline for text cleaning, deduplication, and mapping to ticker
          symbols.
    \item Integrate the FinBERT model to perform financial sentiment inference on text data.
    \item Add a backend endpoint for analyzing pasted text and returning sentiment results.
    \item Add a backend endpoint for analyzing uploaded CSV files and returning sentiment results.
    \item Develop a news ingestion adapter to fetch recent financial headlines for given tickers.
    \item Develop a Reddit/StockTwits ingestion adapter to collect ticker-specific posts.
    \item Create an aggregation module to compute sentiment metrics (mean score, percent positive,
          volume, positive/negative ratio).
    \item Implement an explainability endpoint that returns token-level sentiment highlights for model
          predictions.
    \item Develop REST API endpoints for metrics, posts, explanations, and system health monitoring.
    \item Build the core React dashboard with ticker input, source toggles, KPI displays, sentiment
          chart, and top posts list.
    \item Add frontend flows for paste-text and file upload sentiment analysis.
    \item Implement user experience states such as loading, empty view, and error handling in the
          frontend.
    \item Containerize both backend and frontend using Docker and supply a Compose file for local
          deployment.
    \item (\textbf{Stretch Goal}) Implement a scheduler and caching mechanism to periodically refresh
          and store sentiment results in memory.
    \item (\textbf{Stretch Goal}) Add a persistent storage layer for texts, sentiment scores, and
          aggregated metrics.
    \item (\textbf{Stretch Goal}) Add a live stock price overlay to the sentiment time-series chart in
          the dashboard.
    \item (\textbf{Stretch Goal}) Develop a prototype predictive model that combines sentiment features
          with price baselines for short-term forecasting.
\end{itemize}


\bibliographystyle{acm}
\bibliography{bibliography.bib}

\end{document}
