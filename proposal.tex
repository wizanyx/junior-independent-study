% This is a template for your written document.
%
% To compile using latexmk on the command line, run the following: 
% latexmk -pdf main.tex

\documentclass[12pt]{article}
\usepackage{setspace}
\singlespace
\usepackage[left=1in,right=1in,top=1in,bottom=1in]{geometry}

\title{\textbf{Sentiment Analysis as a tool for Financial Market Analysis}}
\author{Anany Sachan}

\begin{document}

\maketitle

In recent decades, sentiment analysis has emerged as one of the most widely studied applications of 
natural language processing (NLP). At its core, sentiment analysis refers to the automatic detection of 
opinions, attitudes, and emotions expressed in text. Foundational work by Pang, Lee, and Vaithyanathan 
established that classical machine learning methods such as Naïve Bayes, Maximum Entropy, and Support 
Vector Machines could be successfully applied to classify movie reviews as positive or negative 
\cite{10.3115/1118693.1118704}. This research established sentiment classification as a formal 
computational task and inspired a wave of subsequent studies across domains ranging from politics to 
product reviews.

As the field matured, researchers began to move beyond polarity detection toward the identification of 
richer affective states. Mohammad surveyed advances in detecting valence, discrete emotions such as joy, 
anger, or fear, and other affectual categories, while also highlighting issues of bias, fairness, and 
societal consequences \cite{SentimentEmotionSurvey2021}. Around the same time, deep learning models 
began to dominate sentiment analysis tasks. Surveys such as Zhang, Wang, and Liu documented how 
convolutional and recurrent neural networks dramatically improved performance compared to traditional 
approaches, especially when trained on large labeled datasets \cite{10.1002/widm.1253}.

The most significant methodological breakthrough came with the introduction of transformer-based 
language models, particularly BERT and its domain-specific variants. Transformers rely on attention 
mechanisms that capture long-range dependencies in text, enabling much more accurate contextual 
representations than bag-of-words or recurrent architectures. Subsequent surveys and applied studies 
demonstrated that transformer models outperform previous baselines across a wide range of sentiment 
tasks \cite{10.1145/3586075,10.1145/3650215.3650260}. In the financial domain, fine-tuned transformer 
models such as FinBERT have been shown to significantly improve sentiment detection in technical texts 
like earnings reports and analyst briefings \cite{10.1145/3543873.3587605,araci2019finbert}.

The significance of these advances lies in the growing recognition that public sentiment is a powerful 
driver of financial markets. Traditional quantitative models, which rely primarily on historical prices 
and trading volume, fail to account for the behavioral and psychological forces that often dictate 
short-term volatility. Events such as the 2021 GameStop short squeeze—fueled largely by sentiment on 
Reddit's r/wallstreetbets—illustrate the real-world consequences of collective investor mood 
\cite{Desiderio_2025}. Incorporating sentiment signals into financial analysis has the potential to 
improve forecasting, risk management, and decision-making \cite{jiang2025financialsentiment}. Recent 
work has begun to test this hypothesis, showing correlations between sentiment derived from social 
media or news and subsequent asset price movements \cite{10.1109/MCI.2018.2866727}. However, these 
methods remain far from standardized, and questions remain regarding reliability, data bias, and 
robustness across market conditions.

The goal of this project is to design and implement a finance-focused sentiment analysis system that 
builds on the trajectory of prior research while addressing gaps in applied usage. Specifically, I will 
develop a \textbf{Finance Sentiment Dashboard}: a software application that ingests financial text data 
(news headlines, social media posts, or discussion forum content) and outputs sentiment classifications 
and visualizations in near real-time. The system will leverage a transformer-based model fine-tuned for 
financial text, such as FinBERT, to provide ticker-specific sentiment scores and trends.

The proposed software will have three main capabilities. First, users will be able to enter a stock 
ticker symbol, at which point the backend will collect relevant recent text data from configured 
sources. Second, the system will preprocess the data and apply the sentiment model to classify texts 
as positive, negative, or neutral with respect to financial outlook. Third, the frontend will display 
results in a clear, interactive dashboard, including aggregate sentiment metrics, temporal trend charts, 
and confidence estimates.

By combining academic rigor with practical functionality, this project will contribute both to scholarly 
discussions and to applied finance practice. It demonstrates how the evolution of sentiment analysis — 
from early machine learning approaches \cite{10.3115/1118693.1118704}, through emotional nuance 
\cite{SentimentEmotionSurvey2021}, to modern transformer architectures 
\cite{10.1145/3586075,10.1145/3543873.3587605} — can be operationalized into a real-world tool. At the 
same time, it aims to provide users with an accessible interface for exploring how public mood 
influences financial markets, reinforcing the significance of sentiment as both a computational and 
economic phenomenon.

\newpage
\section*{Appendix}
A concise list of features / user stories in the order in which they will be built. A few examples are below to demonstrate the expected scope and level of granularity; you will have more features than this.
\begin{itemize}
	\item Default picture display on web application.
	\item On a button-click, user can separate the image into foreground and background.
	\item User can select a picture from their desktop.
	\item Selected picture displays on the web application.
\end{itemize}


\bibliographystyle{acm}
\bibliography{bibliography.bib}

\end{document}
